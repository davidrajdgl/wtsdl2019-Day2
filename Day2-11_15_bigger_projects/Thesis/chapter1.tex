\chapter{Introduction}
% \chaptermark{\small Positive matrices and its application}
\label{ch:intro}
\section{Definitions}
Conceptually, a graph is a mathematical model that represents the binary relationship among a collection of well defined 
objects. Formally, a {\it graph} $G$ consists of two finite non-empty sets $V$ and $E$. The elements of $V$ are 
known as the vertices of the graph $G$. The set $E$ consists of unordered pairs of distinct vertices and are  
known as the edges of $G$. We use $G(V,E)$ or $G$ to represent a graph $G$ with vertex set $V$ and edge set $E$.  Two 
vertices $u,v$ in $G$ are said to be adjacent if $\{u,v\}\in E(G)$.

The graph $G$ with vertex set $V=\{1,2,3,4,5,6\}$ and edge set $$E=\{\{1,2\},\{2,3\},\{3,4\},\{4,5\},\{5,6\},\{1,6\},\{2,5\}\}$$ 
is shown in the Figure \ref{fig1.0}
\begin{figure}[h!]
 \begin{pspicture}(7,6)
\rput(4,0){\scalebox{1}{
\pscircle[linewidth=.05, linecolor=blue](1,1){.25}
\pscircle[linewidth=.05, linecolor=blue](1,3){.25}
\pscircle[linewidth=.05, linecolor=blue](1,5){.25}
\pscircle[linewidth=.05, linecolor=blue](5,1){.25}
\pscircle[linewidth=.05, linecolor=blue](5,3){.25}
\pscircle[linewidth=.05, linecolor=blue](5,5){.25}


\psline[linewidth=.05](1.2,1.1)(4.9,4.8)
\psline[linewidth=.05](1.2,4.9)(4.9,1.2)
\psline[linewidth=.05](1.2,3)(4.8,3)
\pscurve[linewidth=.05](.8,1)(.5,2)(.8,3)
\pscurve[linewidth=.05](.8,5)(.5,4)(.8,3)
\pscurve[linewidth=.05](5.2,1)(5.5,2)(5.2,3)
\pscurve[linewidth=.05](5.2,5)(5.5,4)(5.2,3)
}
\rput(.95,1){ $3$}
\rput(.95,3){ $2$}
\rput(.95,5){ $1$}
\rput(4.95,5){ $4$}
\rput(4.95,3){ $5$}
\rput(4.95,1){ $6$}
}
\end{pspicture}
\caption{\label{fig1.0} The graph $G$.}
\end{figure}
A walk in a graph $G$ is a sequence of vertices $v_1,v_2,...,v_n$ in which $v_i$ is adjacent 
to 
$v_{i+1}$, for $i=1,2,\cdots,n-1$. 
By a {\it path} in a graph $G$ we mean a sequence of distinct vertices 
$v_1,v_2,\ldots, v_n$ in which $v_i$ is adjacent to
$v_{i+1}$, for $i=1,2\ldots n-1.$
A path in which end vertex and the start vertex are same is called a {\it cycle}. A graph that has a cycle is known as a {\it 
cyclic} graph; otherwise, it is said to be acyclic.
A graph $G$ is said to be {\it connected} if each pair of vertices in a graph $G$ are connected by a path. 
A {\it tree} is a graph that is connected and acyclic. 
For basic terminologies used in this dissertation we refer \cite{har69}.
\begin{figure}[h!]
 \begin{pspicture}(7,6)
\rput(4,0){\scalebox{1}{
\pscircle[linewidth=.05, linecolor=blue](1,1){.25}
\pscircle[linewidth=.05, linecolor=blue](1,3){.25}
\pscircle[linewidth=.05, linecolor=blue](1,5){.25}
\pscircle[linewidth=.05, linecolor=blue](5,1){.25}
\pscircle[linewidth=.05, linecolor=blue](5,3){.25}
\pscircle[linewidth=.05, linecolor=blue](5,5){.25}


\psline[linewidth=.05,linecolor=red](1.2,1.1)(4.9,4.8)
\psline[linewidth=.05](1.2,4.9)(4.9,1.2)
\psline[linewidth=.05,linecolor=red](1.2,3)(4.8,3)
\pscurve[linewidth=.05,linecolor=red](.8,1)(.5,2)(.8,3)
\pscurve[linewidth=.05,linecolor=red](.8,5)(.5,4)(.8,3)
\pscurve[linewidth=.05](5.2,1)(5.5,2)(5.2,3)
\pscurve[linewidth=.05,linecolor=red](5.2,5)(5.5,4)(5.2,3)
}
\rput(.95,1){ $3$}
\rput(.95,3){ $2$}
\rput(.95,5){ $1$}
\rput(4.95,5){ $4$}
\rput(4.95,3){ $5$}
\rput(4.95,1){ $6$}
}
\end{pspicture}
\caption{\label{fig1.1} The graph $G$.}
\end{figure}
The walk $1-2-3-4-5-2-1$ in  the Figure \ref{fig1.1} is represented by the red edges. 
\begin{figure}[h!]
\begin{pspicture}(8,4)
\rput(2,0){\scalebox{1}{
\pscircle[linewidth=.05, linecolor=blue](1,1){.25}
\pscircle[linewidth=.05, linecolor=blue](3,1){.25}
\pscircle[linewidth=.05, linecolor=blue](5,1){.25}
\pscircle[linewidth=.05, linecolor=blue](7,1){.25}
\pscircle[linewidth=.05, linecolor=blue](3,3){.25}


\psline[linewidth=.05, linecolor=green](1.3,1)(2.7,1)
\psline[linewidth=.05](3.3,1)(4.7,1)
\psline[linewidth=.05, linecolor=green](5.3,1)(6.7,1)
\psline[linewidth=.05, linecolor=green](3,1.3)(3,2.7)
\psline[linewidth=.05](3.2,2.8)(6.8,1.2)
\psline[linewidth=.05, linecolor=green](3.2,2.8)(4.8,1.2)
\rput(.95,1){ $1$}
\rput(3,1){ $2$}
\rput(5,1){ $3$}
\rput(7,1){ $4$}
\rput(3,3){ $5$}
}}
\end{pspicture}
\caption{\label{fig3.0} }
\end{figure}
The green line in  the Figure \ref{fig3.0} represents the path $1-2-5-3-4$.
\begin{figure}[h!]
\begin{pspicture}(8,4)
\rput(2,0){\scalebox{1}{
\pscircle[linewidth=.05, linecolor=blue](1,1){.25}
\pscircle[linewidth=.05, linecolor=blue](3,1){.25}
\pscircle[linewidth=.05, linecolor=blue](5,1){.25}
\pscircle[linewidth=.05, linecolor=blue](7,1){.25}
\pscircle[linewidth=.05, linecolor=blue](3,3){.25}
\psline[linewidth=.05](1.3,1)(2.7,1)
\psline[linewidth=.05, linecolor=blue](3.3,1)(4.7,1)
\psline[linewidth=.05](5.3,1)(6.7,1)
\psline[linewidth=.05, linecolor=blue](3,1.3)(3,2.7)
\psline[linewidth=.05](3.2,2.8)(6.8,1.2)
\psline[linewidth=.05, linecolor=blue](3.2,2.8)(4.8,1.2)
\rput(.95,1){ $1$}
\rput(3,1){ $2$}
\rput(5,1){ $3$}
\rput(7,1){ $4$}
\rput(3,3){ $5$}
}}
\end{pspicture}
\caption{\label{fig4.0} }
\end{figure}
The blue line in  the Figure \ref{fig4.0} represents the cycle $2-5-3-2$.
\section{Basic Results}

\begin{theorem}
 Total degree of the vertices of any graph $G$ is twice the number of edges in it.
\end{theorem}\begin{theorem}
 The number of odd vertices in any graph $G$ is even.
\end{theorem}
\begin{theorem}
 A connected graph on $p$-vertices and $q$-edges with $q\geq p$ contains a cycle of length at least $\delta(G)-1$, where 
$\delta(G)$ is the minimum degree of the graph.
\end{theorem}

\begin{theorem}
 Let $G$ be a graph with $p$ vertices and $q$ edges. The following are equivalent:
 \begin{enumerate}[(a).]
  \item $G$ is a tree.
  \item $G$ is connected and $q=p-1$.
  \item $G$ is acyclic and $q=p-1$.
 \end{enumerate}
\end{theorem}
A graph that has exactly one cycle is known as an {\it uni-cyclic graph}. 
\begin{theorem}
 A connected $(p,q)$-graph is unicyclic if and only if $q=p$.
 \end{theorem} 
\begin{proof}
  Let $G=(p,q)$ be a connected graph.
First suppose that $G=(p,q)$ is unicyclic. Then for any cyclic edge $e$ of $G$, the graph $G-e$ is a tree and so 
$q(G-e)=p(G-e)-1$ i.e. $q-1=p-1$ i.e. $q=p.$

Next assume that $q=p$. We need to show that $G$ is unicyclic.
Let $G=(p,q)$ be acyclic. Since $G=(p,q)$ is connected and acyclic, so it is a tree and hene $q=p-1$, which is a 
contradiction. Therefore $G$ is unicyclic. Also, for any cyclic edge $e$ of $G$, the graph $G-e$ is connected and $q=p-1$ i.e. 
$G-e$ 
is acyclic. Hence, removal of any cyclic edge makes the graph $G$ acyclic. Therefore $G$ has exactly one cycle or $G$ is 
unicyclic.
\end{proof}
\section{Brief Review of the Literature}\label{lit}
The concept of  signed graphs and balance signed graphs 
 were introduced by Frank Harary  to treat a question in 
social psychology\cite{HAR53}.
It is 
remarkable that years before Harary, Konig \cite{DEN50} had the idea of a graph with a distinguished subset of edges, in a way we 
now recognize as equivalent to signed graphs and even proved Harary's balance theorem and defined switching in the form of taking 
the sum of a cut-set  with negative edges.
Data in the social sciences can often be modeled using a signed graph, a graph where every edge has a sign $+$ or $-$,or a marked 
graph, a graph where every vertex has a sign $+$ or $-$. A marked graph is called consistent if every cycle has an even number 
of vertices with $-$ sign. The concept of consistency is analogous to the concept of balance in signed graphs: A signed graph is 
called balanced if every cycle has an even number of edges with $-$ sign. 

Given signed graph $G$, a marking of $G$ induces naturally as follows: sign of a vertex $v\in V(G)$ is the product of the sign of 
the edges incident with it. This marking of a signed graph $G$ is known as canonical marking. A signed graph that is consistent 
with respect to its canonical marking is known as a canonically consistent signed graph. 

Many results related to characterization of consistent marked graphs are available in the literature. Beineke  and Harary 
\cite{BEI1978, BEI78} were 
the  first  to  pose  the  problem  of  characterizing  consistent  marked  graphs,  
which  was  eventually  settled  independently  by  Acharya  in \cite{ACH83, ACH84},  Hoede in \cite{HOE92}  
and  Rao in \cite{RAO84}. In \cite{HOE92} it was shown that a marked graph is consistent if and only, for any spanning tree $T$ 
all fundamental cycles are positive and all common paths of pairs of fundamental cycles have end points with the same marking.  
Further characterizations  of  consistent  marked  graphs  
have  been  obtained  by  Roberts  and  Xu in \cite{ROB03, ROB95}.  A 
characterization of canonically consistent total signed graphs is discussed in \cite{SIN13}.

In this dissertation, we have 
discussed the characterization of canonically consistent 2-path signed graphs and 2-path product signed graphs.  is discussed in 
We  also tried to 
address the following questions:

Let $G$ be any signed graph and $G\# G$ be the 2-path signed graph of $G$.
Define $$S=\{G~|~G \text{ is balanced and canonically consistent}\}$$
$$S_1=\{G\in S~|~G\# G \text{ is balanced and canonically consistent}\}$$
$$S_2=\{G\in S~|~G\# G \text{ is balanced but not canonically consistent}\}$$
$$S_3=\{G\in S~|~G\# G \text{ is canonically consistent but not balanced}\}$$
$$S_4=\{G\in S~|~G\# G \text{ is neither canonically consistent nor balanced}\}$$
\begin{enumerate}
 \item Is $S_1=\Phi$?
  \item Is $S_2=\Phi$?
   \item Is $S_3=\Phi$?
    \item Is $S_4=\Phi$?
    \item If $S_1\neq\Phi$ then under what conditions $G\in S\Rightarrow G\in S_1$?
    \item  If $S_2\neq\Phi$ then under what condition $G\in S\Rightarrow G\in S_2$?
    \item If $S_3\neq\Phi$ then under what condition $G\in S\Rightarrow G\in S_3$?
    \item  If $S_4\neq\Phi$ then under what condition $G\in S\Rightarrow G\in S_4$?
\end{enumerate}


